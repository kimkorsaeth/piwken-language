\documentclass[smallroyalvopaper,9pt]{memoir}
\usepackage{multicol, multirow, array} % table and column pages formatting
\usepackage{fontspec} % font selection
\usepackage{anyfontsize} % font sizes
\usepackage[hidelinks]{hyperref} % links to different sections of pdf
\usepackage{url}        % url formatting
\usepackage{amsmath}    % for the substack function in RBPN
\usepackage{float}      % floats 
\usepackage{graphicx}   % for pictures
\usepackage{booktabs}   % nicer tab lines
\usepackage{tabto}      % for tabbing
\usepackage{qtree}      % for trees, particularly in phonology section
\usepackage{expex}      % glossing package
\usepackage{phonrule}   % phonotactics rules
\usepackage{enumitem}   % lists (enumerate)
\usepackage[calc,english]{datetime2} % auto updates date

%-----CONFIGURATION------
%------------------------

\settrimmedsize{234mm}{156mm}{*}

\setmainfont{STIX Two Text}
\restylefloat{table}

\setsecnumdepth{subsubsection}
\settocdepth{subsubsection}

\lingset{
    glstyle=nlevel,
    numoffset=1em,
    textoffset=1em,
    exskip=.75ex,
    belowglpreambleskip=.25ex,
    aboveglftskip=.25ex,
    interpartskip=4ex
}

\DTMnewdatestyle{eurodate}{%
    \renewcommand{\DTMdisplaydate}[4]{%
        \number##3.\nobreakspace%           day
        \DTMmonthname{##2}\nobreakspace%    month
        \number##1%                         year
    }%
    \renewcommand{\DTMDisplaydate}{\DTMdisplaydate}%
}

\DTMsetdatestyle{eurodate}

\renewcommand{\arraystretch}{1.5}
\setlength{\tabcolsep}{4pt}
\setlength\columnseprule{0.5pt}

\lingset{belowpreambleskip=2ex, interpartskip=4ex}

\NumTabs{10}

%-------COMMANDS---------
%------------------------

\newcommand{\lang}{Piwkeneth}
\newcommand{\langfr}{Nabasais}
\newcommand{\langeng}{Piwken}
\newcommand{\people}{TBD}

\newcommand{\longv}{ː}
\newcommand{\sqbrack}[1]{$\langle$#1$\rangle$}
\newcommand{\ttilde}{\raise.17ex\hbox{$\scriptstyle\sim$}}
\newcommand{\bind}{\symbol{"0361}}
\newcommand{\glem}[1]{\underline{\smash{#1}}}
\newcommand{\tsub}[1]{\textsubscript{#1}}
\newcommand{\unv}[1]{#1\symbol{"325}}

%-------TITLE PAGE-------
%------------------------

\title{
    \fontsize{70}{60}\selectfont 
    {the \langeng{} language} \\
    \sffamily 
    \vspace{2cm}
    \fontsize{30}{32}\selectfont 
    Reference Grammar\\
    \vspace{2cm}
    Kim Korsæth
}
\author{}
\date{\today}

%----TABLE OF CONTENTS---
%------------------------

\cftpagenumbersoff{part}
\cftsetindents{part}{1.5em}{1.5em}
\renewcommand\cftchapterfont{\textrm}



%--------MAIN DOC--------
%------------------------

\begin{document}


\maketitle

\newpage

\frontmatter

\chapter*{Preface}

\lang{} is a constructed language, and the speakers are entirely fictional. This grammar is a work of fiction.

THIS IS A WORK IN PROGRESS. Some sections may lack a lot of detail or be worded very poorly. I'm not good at translating my thoughts and ideas into words, and while I'm constantly refining and expanding the whole grammar, it'll be at my own (considerably slower) pace. I ask that you reserve some judgement until I can call this done, and hope that you can enjoy and get excited by what is here so far.

\newpage
\tableofcontents
\listoffigures
\listoftables
\newpage

\chapter{Abbreviations}

\begin{multicols*}{2}
    
\newcommand{\listabbrev}[2]{
\begin{minipage}[t]{0.17\columnwidth}
    \textbf{#1}
\end{minipage}
\begin{minipage}[t]{0.8\columnwidth}
    {#2}
\end{minipage}
\vspace{1ex}
}

\noindent\listabbrev{*}{unattested, ungrammatical, or unacceptable}
\listabbrev{›}{acts upon, e.g. 1›2, 3›3'}
\listabbrev{\textsc{act}}{actual aspect}
\listabbrev{\textsc{ani}}{animate}
\listabbrev{\textsc{antip}}{antipassive}
\listabbrev{C}{consonant}
\listabbrev{\textsc{der}}{derivational morpheme}
\listabbrev{\textsc{du}}{dual number}
\listabbrev{\textsc{fact}}{factual aspect}
\listabbrev{\textsc{hab}}{habitual aspect}
\listabbrev{\textsc{ina}}{inanimate}
\listabbrev{\textsc{inst}}{instrumental}
\listabbrev{\textsc{nact}}{non-actual aspect}
\listabbrev{\textsc{o, obj}}{object}
\listabbrev{\textsc{obv, '}}{obviative, e.g. 3'}
\listabbrev{\textsc{p, pl}}{plural number}
\listabbrev{\textsc{pct}}{punctual aspect}
\listabbrev{\textsc{prt}}{partitive}
\listabbrev{\textsc{s, sg}}{singular number}
\listabbrev{\textsc{sbj}}{subjunctive}
\listabbrev{\textsc{V}}{vowel}
    
\end{multicols*}

\newpage

\mainmatter

\chapter{The Piwken people}

Piwkenaisetwem, the Piwken people of Manitoba, have many hundreds of years of history in the area. Nowadays, they're mostly found in the Piwken First Nation on the border of Lake Winnipeg. While there are about 1800 native Piwken living in and around the reservation, only about 300 of them still speak their native language, all above the age of 50, and almost none of them are monolingual speakers. It's estimated that if no action is taken to preserve the language, it may become extinct within the next 40 years. This grammar and its accompanying dictionary has been developed in cooperation with the Piwken Council as the basis for developing learning material and other media to revitalize the language. With more children being immersed from an early age and taught the importance of preserving their culture, the Piwkenaisetwem hope to see their language survive for many generations yet.

\chapter{Phonology}

Unless one takes precautions, it is easy to naïvely approach new languages with a Western (Indo-European) mindset and presuppositions about their phonological frameworks. In order to fairly and accurately assess \langeng{}, we must try to eliminate as many biases as posible and build a new framework from the ground up.

The phonemic inventory of \langeng{}, at 14 contrasting segments, is considerably smaller than the world-average 27. While there is decent overlap of /i u/ and /j w/ owing to their shared underlying composition, it still makes sense to analyze them as different phonemes because there are environments where they can only be analyzed as one set as opposed to the other, or the environment favors one over other even if both could theoretically apply.

The French grammar had a very simple and inaccurate phonological analysis where the voiced allophones of /p t k/ were considered a discrete set of phonemes, /ɬ/ was referred to as /ʃ/, 

\section{Consonants}

\begin{table}[ht]
    \centering
    \begin{tabular}{rccccc}
        \toprule
        & Labial & Alveolar & Palatal & Velar & Glottal \\
        \midrule
        Plosive & p & t & & k & \\
        Fricative & & s & & & h \\
        Lateral fricative & & ɬ & & & \\
        Nasal & m & n & & & \\
        Approximant & & & j & w & \\
        \bottomrule
    \end{tabular}
    \caption{Phonemic consonants of \lang}
\end{table}

% ɡ unicameral g here

\lang{} distinguishes 10 consonant phonemes.

% put here explanations for each phoneme along with initial, medial and final realizations along with any other special cases.

\paragraph{/p/} is a bilabial plosive. It is realized as an unaspirated [p], or [b] if intervocalic. Word-final /p/ often becomes [ʔ].

\paragraph{/t/} is an alveolar plosive. It is realized as an unaspirated apical [t], or [d] if intervocalic.

\paragraph{/k/} is a velar plosive. It is realized as an unaspirated [k], or [g] if intervocalic. Word-final /k/ becomes [x]. Like many velars cross-linguistically, the articulation is quite weak, and can be affricated, fricated, or dropped in many cases. Intervocalic /k/ has a tendency to be elided all the way down to [ɣ] intervocalically.

\paragraph{/s/} is an alveolar fricative. It is realized as an apical [s], or [z] if intervocalic. Word-initially, it becomes [t\bind{}s].

\paragraph{/ɬ/} is a lateral alveolar fricative. It is realized as [ɬ] with the tip of the tongue between the front teeth and the articulation occuring further back on the tongue as a result, but still on the velar ridge. Before /i/ it becomes [θ], or [ç] if preceded by [k]. Word-initially, it becomes [t\bind{}ɬ].

\paragraph{/h/} is a glottal fricative. It is for the most part realized as [h], but can become a strongly fricated [x] after plosives. Word-initially it becomes [ʔ].

\paragraph{/m/} is a bilabial nasal, always realized as [m].

\paragraph{/n/} is an alveolar nasal. It is realized as [n], but can assimilate with adjacent consonants to become [m] or [ŋ]. 

\paragraph{/j/} is a palatal approximant. 

\paragraph{/w/} is a labiovelar approximant.

\subsection{Intervocalic voicing} \label{intervocalic}

It is a feature of all unvoiced segments except /ɬ/ and /h/ to become voiced between two voiced segments. The voiced segments must be directly adjacent, and clusters of unvoiced consonants are never voiced.

% examples here

\subsection{Initial fortition} \label{initfort}

The fricatives /s h ł/ undergo fortition to [t\bind{}s ʔ t\bind{}ɬ] word-initially.

% examples here

\subsection{Geminization}

Sequences of the same consonant are geminated and produced as a single release.

\section{Vowels}

Vowels are defined by a set of phonological restrictions that do not apply to consonants; likewise, consonants can in turn be defined by their non-vowelness.

\begin{table}[ht]
    \centering
    \begin{tabular}{lll}
        \toprule
        Rule & Excludes \\
        \midrule
        Vowels must be continuants & /p t k/ \\
        Vowels are always voiced & /s h ɬ/ \\
        Vowels are oral & /m n/ \\
        Vowels can't break long vowel sequences & /j w/ \\
        \bottomrule
    \end{tabular}
\end{table}

The above rules leave us with only the sequences that behave in a way that we can traditionally categorize as vowels.

\begin{table}[ht]
    \centering
    \begin{tabular}{lccc}
        \toprule
        & Front & Central & Back \\
        \midrule
        High & i & & u \\
        Mid & e & &  \\
        Low & & a & \\
        \bottomrule
    \end{tabular}
    \caption{Phonemic vowel inventory of \lang}
\end{table}

\lang{} contrasts only four vowel qualities. 

\paragraph{/a/} is a low central vowel, almost always pronounced like the cardinal vowel [a]. 

\paragraph{/e/} is a mid front vowel, usually pronounced like [e], but may be lowered to [ɛ]. Before /w/, it is realized as a schwa [ə].

\paragraph{/i/} is a high front vowel, pronounced like the cardinal vowel [i].

\paragraph{/u/} is a high back rounded vowel, usually pronounced like a near-back vowel [ʊ]. It can sometimes appear as low as [o].

\subsection{Diphthongs and long vowels}

Up to two vowels may occur in sequence, and are realized as one consistent glide from one vowel quality to the other. No vowel is given more weight as the "main vowel". This same process creates long vowels when two identical vowels occur in sequence.

\section{Phonotactics}

We can't conclusively demonstrate the existence of a syllable structure in \langeng{}, however the phonotactic rules often produce results that resemble syllables. Moras seem to be a more apt concept to describe these rules, and speakers seem to intuitively divide words into moras. The established phonotactic rules are outlined below:

\begin{enumerate}
    \itemsep0em 
    \item a mora can comprise a vowel or consonant-vowel sequence.
    \item a mora can also comprise just a consonant if it divides a vowel and a non-vowel (i.e. consonant or word boundary).
    \item an otherwise unbound consonant preceding a vowel must combine to form one mora.
    \item only runs of up to 2 vowels or consonants is allowed; any third V or C in a row must be separated with an epenthetic segment.
    \item words *must* comprise at least a two-segment sequence that is either one or two moras.
\end{enumerate}

% legal sequences so far (these can overlap):
% \begin{itemize}
%     \itemsep0em 
%     \item V - i, u, a, e
%     \item CV - ki, tu, pa, ne etc.
%     \item VC - et, um, ap etc.
%     \item VV - ai, e: etc.
%     \item CVC - nah, mit, kup etc.
%     \item CVV - na:, nai, nae, nau etc.
%     \item VCV - ata, iku etc.
%     \item VCCV - ahki, edma etc.    
% \end{itemize}

% illegal sequences (\% word boundary):
% \begin{itemize}
%     \itemsep0em 
%     \item \%C\%
%     \item \%CCV
%     \item VCC\%
%     \item CCC
%     \item VVV    
% \end{itemize}

The 1st/2nd person suffix -s can appear word-finally after one consonant, violating phonotactics.

There doesn't seem to be any restrictions in terms of which consonants can appear where.

/ju/ and /iw/ are in complementary distribution, same as /uj/ and /wi/.

/wa/ has a tendency to be realized as [o] in fast speech.

/Vji/ is rare and very frequently turns into /Vj/; /\%ijV/ into /\%jV/ has already happened recently.

\newpage

\section{Orthography}

\begin{table}[ht]
    \centering
    \begin{tabular}{llll}
        \toprule
        Letter & Phoneme & Letter & Phoneme\\
        \midrule
        a & /a/ & n & /n/ \\
        e & /e\ttilde{}ə/ & p & /p/ \\
        h & /h/ & s & /s/ \\
        i & /i/ & t & /t/ \\
        k & /k/ & u & /u/ \\
        th& /ɬ/ & w & /w/ \\
        m & /m/ & y & /j/ \\
        \bottomrule
    \end{tabular}
    \caption{\lang{} orthography, devised by \$\$\$}
\end{table}

There have been efforts to adopt Cree syllabics, but this has so far not seen widespread support.

\chapter{Verbs} 

% TOTALLY retain the move/feel/change/remain distinction in some way 

The verb is the biggest focal point of \lang{} morphosyntax. Most things are expressed within the verb phrase. The distinction between inflection and derivation is difficult to ascertain, and a lot of affixes can be either.

Since verbs can be derived from both verbal and nominal roots, no distinction is made between the two in this section. Instead, I will refer to stems as the nucleus of the morphological verb throughout this chapter.

The verbal morphology can be laid out as such:

\begin{table}[ht]
    \centering
    \begin{tabular}{>{\bfseries}ll}
        \toprule
        Slot & Use \\
        \midrule
        -1 & preverbal affix \\
        0 & root \\
        1 & reduplication \\
        2 & additional verb \\
        3 & derivational suffixes \\
        4 & -y suffix \\
        5 & final suffix \\
        \bottomrule
    \end{tabular}
\end{table}

\section{Verb classes}

Verbs are divided into four categories depending on factors like the number and types of arguments, the lexical content, and morphology. These factors come from the verb stem.

\subsection{\textsc{move} class}

Verbs in the \textsc{move} class describe some kind of physical movement from one place or orientation to another, either of its own power or by means of an external volitive or natural force. 

\subsection{\textsc{change} class}

Verbs in the \textsc{change} class describe some kind of physical alteration of an entity. This involves compositional, deformational, or aesthetic changes, which can be either momentary or lasting. Prototypical \textsc{change}-type verbs involve either one entity retaining its physical composition but is deformed into another shape (e.g. breaking or bending a stick) or one entity transforming into something else as a whole (e.g. water freezing into ice).

\subsection{\textsc{feel} class}

Verbs in the \textsc{feel} class describe actions or events with no clear physical force imparted on someone or something. This includes sensory impressions, feelings, thoughts, talking, crying, and so on.

\subsection{\textsc{remain}-type}

Verbs in the \textsc{remain} class are stative verbs describing states, qualities, characteristics.

\section{Preverbal affixes}

A group of prefixes come before the verb stem to mark various modal and aspectual things. There are two sets, and the set used for a given verb depends on factors like verb class and 

\begin{table}[ht]
    \centering
    \begin{tabular}{lcc}
        \toprule
                   & Set 1  & Set 2 \\
        \midrule
        Factual    & i-     & ki- \\
        Nonactual  & lea-   & kia- \\
        Actual     & a-     & ka- \\
        Punctual   & ?-     & ?- \\
        Negative   & ?-     & ?- \\
        Defective  & ?      & ?- \\
        Jussive    & ?-     & ?- \\
        Resumptive & ?-     & ?- \\
        \bottomrule
    \end{tabular}
    \caption{Preverbs of \lang}
\end{table}

Each verb class uses a particular set of preverbs:

\begin{table}[ht]
    \centering
    \begin{tabular}{>{\sc}lcc}
        \toprule
        & Set 1 & Set 2 \\
        \midrule
        move & cislocative motion & translocative motion \\
        change & X & - \\
        feel & - & X \\
        remain & X* & -\\
        \bottomrule
    \end{tabular}
    \caption{Preverb sets used with each verb class}
\end{table}

The \textsc{move} verb class uses either set depending on the motion described. Any movement that starts and finishes in the vicinity of the relevant area is called cislocative motion and takes set 1. Any movement that starts nearby and finishes somewhere that's far away or not relevant, or both starts and finishes in an area far away or unseen by the discourse participants is called translocative motion and takes set 2.

The \textsc{change} verb class uses set 1.

The \textsc{feel} verb class uses set 2.

The \textsc{remain} verb class can only use the Factual, Nonactual, Negative, and Jussive preverbs from set 1.

\subsection{Factual preverbs}

"indicative" for stative stems (nouns, adjectives)

habitual for active stems (durative, punctual)

\subsection{Nonactual preverbs}

irrealis/subjunctive-ish, verbal noun-ish

The Nonactual preverbs are used to mark that an action or event being talked about as non-finite; it is not currently happening, and the discourse is rather about the concept of the action itself. This differs from the Negative preverbs, which assert the untruth of a statement. It can in many ways be likened to English infinitives and gerunds, but also subjunctive verbs (i.e. 'I \textit{would} [...]'). 

\subsection{Actual preverbs}

"indicative" for durative stems

frequentative takes Actual preverb

The actual preverbs mark an action or event happening at an indicated time. They are closer to a realis mood marker as it does not encode any tense or aspect. They contrast most directly with the Nonactual preverbs.

\pex
\a\begingl
ka-[\textsc{act.ii}-]@
hahu[sleep]
\glft `they are sleeping'
\endgl
\a\begingl
kia-[\textsc{nact.ii}-]@
hahu[sleep]
\glft `sleeping', `they would sleep'
\endgl
\xe

\subsection{Negative preverbs}

negation!

\subsection{Punctual preverbs}

"indicative" for punctual stems

\subsection{Defective preverbs}

action halted before culmination, either unstarted (atelic) or unfinished (telic)

\subsection{Jussive preverbs}

should, ought to, have to, must

\subsection{Resumptive preverbs}

once more, resumes, repeats (once), NOT frequentative

\section{Verb stem}

\subsection{Punctual stems}

instantaneous events or actions

\subsection{Durative stems}

lasting events or actions

\subsection{Stative stems}

States, description, copula-ish

\subsection{Inchoative stems}

start of durative event

\subsection{Frequentative stems}

repetitive action

\subsection{Reduplication}

Two patterns of reduplication have been recorded, rightward duplication and full reduplication. The latter is more productive than the former.

\subsubsection{Rightward reduplication}

Rightward reduplication involves copying some part of the root onto the end of the root, and can take several different forms depending on the root's final syllable or syllables, shown in the table below.

\begin{table}[ht]
    \centering
    \begin{tabular}{lll}
        \toprule
        Root ending & Reduplicated & Example \\
        \midrule
        CV    & CVC(V)  & -pap `become large, grow' (< -pa `be large') \\
              &         & -wanin `deepen, sink' (< -wani `be deep') \\
              &         & -hahuhu `fall asleep' (< -hahu `sleep') \\
        (C)VC & CVCV    & -kutu `about to notice' (< -kut `notice') \\
              &         & -hepasa `yield, buckle' (< -hepas `bend') \\
              &         & -newe `where to' (< -new `what') \\
        CV\longv{} & CV\longv{}CV & -tiidi `sit down' (< -tii `sit, be sitting') \\
              &         & -taata `split in two' (< -taa `two, a pair') \\
              &         & -thameeme `go around' (< -thamee `return') \\
        (C)VCCV & (C)VCCVCV & -patwewe `set off' (< -patwe `go') \\
              &         & -untutu `demonstrate' (< -untu `lead') \\
        \bottomrule
    \end{tabular}
    \caption{Rightward reduplication patterns in \lang}
\end{table}

All the different patterns have in common that they only add one mora to the stem. The simplest process is for light monosyllables, where they take a duplicate of the onset consonant as a coda, or duplicate the entire last syllable if that onset is /h/. On closed syllables, the nucleus and coda are duplicate to form another syllable with the same coda. On long vowel syllables, the first mora of the syllable is duplicated. On multisyllabic roots ending in a light syllable preceded by a heavy syllable, the entire last syllable is reduplicated.

All stems formed by rightward reduplication fall into the category of inchoative stems, as they all describe some form of transition into a new state or action. On stative stems it marks meanings akin to `become X' or `turn into X', on durative stems it marks a meaning akin to `start to X', while on stems describing punctual events it suggests imminent action akin to `about to X'. The resulting inchoative stems usually end up punctual in nature, taking the punctual preverb. Many inchoative stems are no longer productive and have taken on meanings that aren't interchangeable with their equivalent underived stems; a couple examples can be seen in the above table.

The exact lexical changes can vary. Some stems simply describe the initiaton of an activity: -kiiki `start to write' (< -kii `write'), -nuipapa `start to perform' (< -nuipa `perform a song/dance'), -tawawa `start to eat' (< -tawa `eat'), -apasas `set off on foot' (< -apas `travel far on foot, trek'). Other stems indicate imminent action: -kutut `about to notice' (< -kut `notice'), -matat `about to shriek' (< -mat `shriek'), -sayai `about to play with' (< -sai `play with'). These stems usually come from punctual actions, but some durative actions can take on a sense closer to this than the simple inchoative.

Others again have more complex relations. -hepasas `bend, yield, give out' implies a permanent or lasting change to the shape of a rigid object, while the original stem -hepas `bend' can be used for any non-permanent or reversible bending of a rigid or jointed object. -thameeme `go around, revolve around' has an adjacent meaning to -thamee `return'; it's understood to come from having been used in the sense of marking the point where a round trip turns back towards the starting location, akin to a thrown object reaching the apex of its arch.

\subsubsection{Full Reduplication}

Full reduplication of the root forms a frequentative stem. This involves duplicating the entire root with a ligature vowel if necessary to be phonotactically valid. On durative stems it marks that the action is either repeated or lengthened. No distinction is made between starting a task several times or continuing one task.

% example here

In contrast, on punctual actions it marks only repetition. 

% example here

Reduplicated stative stems translate to something like `just X, simply X' or in some cases `barely X'. In almost all cases, whatever is marked undershoots expectations in some way, to the surprise or relief of the relevant party, rarely disappointment.

% example here

This can also be used for affection or adoration towards someone or something.

% example here

\subsection{Additional verb}

You can add another verb to the stem. the extra verb is usually about posture or movement. Oftentimes has an imperfective connotation but can also be read literally.

\subsection{Derivational affixes}

These affixes 

\begin{table}[ht]
    \centering
    \begin{tabular}{ll}
        \toprule
        -hi & 'entirely' or 'all of' \\
        -at & 'back and forth', 'there and back' \\
        -nen & 'one after another', 'one another' \\
        -ke & 'come and X' \\
        -yeets & 'make use of X' \\
        \bottomrule
    \end{tabular}
    \caption{List of verbal derivational affixes}
\end{table}

\subsubsection{-hi 'entirely', 'all of'}

\subsubsection{-at 'back and forth'}

\subsubsection{-nen 'one after another'}

\subsubsection{-ke 'come and X'}

\subsubsection{-yeets 'make use of X'}

\section{Final suffix}

\begin{table}[ht]
    \centering
    \begin{tabular}{lcccc}
        \toprule
         & \multicolumn{4}{c}{Following...} \\
         & C & CC & V & VV \\
        \midrule
        1st and 2nd person & -s & -ih & -s & -s \\
        3rd person animate & -Ce & -e & -e & -ne \\
        3rd person inanimate & -Ca & -a & -e & -ne \\
        \bottomrule
    \end{tabular}
\end{table}

The third person affixes following C replicate the consonant they follow to make them longer.

Intransitive verbs take no person marking. Transitive verbs are marked for person agreeing with the syntactic pivot. The first person is used for the speaker or a group including the speaker. The second person is used for the addressee (whoever is being spoken to) in much the same way. The third person refers to anything not a speech act participant, including impersonal or circumstantial events, e.g. `it's raining'.

\chapter{Nominals}

While pure nouns exist, it is very common for nominals to be derived from verbs.

\section{Case}

\subsection{Absolutive case}

least agentive argument takes absolutive, always. 

intransitive verbs take absolutive because the sole argument is the least agentive argument.

\subsection{Ergative 1 case}

marks the agent of transitive verbs where the action/change described can only come about as a result of someone's action. e.g. something can't be chewed by any other means than someone doing it, making the agent inherent to the action, and so the agent of the verb 'chew' takes the ergative 1 case. 

\subsection{Ergative 2 case}

% can also maybe be called Causative?

marks the agent of transitive verbs where the action/change described can happen through various means, not just through deliberate action. e.g. something can fall apart on its own, or by inanimate forces of nature, but a volitive agent can come around and take it apart deliberately. Similarly, something can heat up on its own, or through external forces both volitive and non-volitive. As such, "take apart" and "warm up(caus.)" would take an ergative 2 agent

\subsection{Dative case}

used for experiencers of \textsc{feel} verb class. Also benefactive and purposive (in order to [verb], for [verb])

\section{Number}

after CC or V, -um; after C VV -wem

\section{Pronouns}

There are personal pronouns but no demonstrative pronouns.

\subsection{Personal pronouns}

\langeng{} does not have third person pronouns, nor does it have any demonstrative pronouns. 

\begin{table}[ht]
    \centering
    \begin{tabular}{lllll}
        \toprule
                             & Absolutive & Ergative 1 & Ergative 2 & Dative \\
        \midrule
        1st person sg.       & ta         & tap      &   & tauk \\
        1st person pl.       & taum       & tapwem   &   & taukwem \\
        1-2. person incl.    & awattaa    & awattaap &   & awattaawuk \\
        2nd person sg.       & igaw       & igaup    &   & igawuk \\
        2nd person pl.       & igawwem    & igaupwem &   & igawukwem \\
        \bottomrule
    \end{tabular}
    \caption{Personal pronouns of \lang}
\end{table}

\section{Demonstrative determiner}

one formative, put before NP to mean "this" or after to mean "that"

In order to work pronominally, must modify [word for "particular one", figure out]

\chapter{Syntax}


\section{The marked argument - Subject}

Always one marked argument. The marked argument is what the utterance is about, and once pragmatically established can be omitted from subsequent utterances. 

Intransitive verbs: one argument, it's automatically the marked argument

Transitive verbs: two arguments, least agentive argument P (patient) is the marked argument. -y on the verb switches this and makes most agentive argument A (agent) the marked argument

Ditransitive verbs: three arguments, least agentive argument T (theme) is the marked argument. -y on the verb switches this and makes more agentive argument R (recipient) the marked argument

Only the marked argument can do certain things?

\section{Simple clauses}

Verbs are always put last in the sentence as a rule. 

\subsection{Intransitive}

subject-verb or null-verb (impersonal)

\subsection{Transitive}

verb last, marked argument first

\subsection{Ditransitive}

theme-recipient-agent-verb for \textsc{give} verbs, others depend

\section{Parataxis}

things are just put next to each other for the most part, subordination does happen but not as often as expected

\end{document}