\documentclass[executivepaper,10pt,twoside,openany,draft]{memoir}
\usepackage{multicol, multirow, array} % table and column pages formatting
\usepackage{fontspec} % font selection
\usepackage{anyfontsize} % font sizes
\usepackage[hidelinks]{hyperref} % links to different sections of pdf
\usepackage{url}        % url formatting
\usepackage{float}      % floats 
\usepackage{graphicx}   % for pictures
\usepackage{booktabs}   % nicer tab lines
\usepackage{tabto}      % for tabbing
\usepackage{textcomp}   % ?? I'm afraid to remove it
\usepackage{qtree}      % for trees, particularly in phonology section
\usepackage{expex}      % glossing package
\usepackage{phonrule}   % phonotactics rules
\usepackage{enumitem}   % lists (enumerate)
\usepackage[calc,english]{datetime2} % auto updates date

%-----CONFIGURATION------
%------------------------

\setmainfont{STIX Two Text}
\restylefloat{table}

\setsecnumdepth{subsubsection}
\settocdepth{subsubsection}

\lingset{
    glstyle=nlevel,
    numoffset=1em,
    textoffset=1em,
    exskip=.75ex,
    belowglpreambleskip=.25ex,
    aboveglftskip=.25ex,
    interpartskip=4ex
}

\DTMnewdatestyle{eurodate}{%
    \renewcommand{\DTMdisplaydate}[4]{%
        \number##3.\nobreakspace%           day
        \DTMmonthname{##2}\nobreakspace%    month
        \number##1%                         year
    }%
    \renewcommand{\DTMDisplaydate}{\DTMdisplaydate}%
}

\DTMsetdatestyle{eurodate}

\renewcommand{\arraystretch}{1.5}
\setlength{\tabcolsep}{4pt}
\setlength\columnseprule{0.5pt}

\lingset{belowpreambleskip=2ex, interpartskip=4ex}

\NumTabs{10}

%-------COMMANDS---------
%------------------------

\newcommand{\lang}{Piwkeneth}
\newcommand{\langfr}{Nabasais}
\newcommand{\langeng}{Piwken}
\newcommand{\people}{TBD}

\newcommand{\longv}{ː}
\newcommand{\sqbrack}[1]{$\langle$#1$\rangle$}
\newcommand{\ttilde}{\raise.17ex\hbox{$\scriptstyle\sim$}}
\newcommand{\bind}{\symbol{"0361}}
\newcommand{\glem}[1]{\underline{\smash{#1}}}
\newcommand{\tsub}[1]{\textsubscript{#1}}
\newcommand{\unv}[1]{#1\symbol{"325}}

%-------TITLE PAGE-------
%------------------------

\title{
    \fontsize{70}{60}\selectfont 
    {the \langeng{} language} \\
    \sffamily 
    \vspace{2cm}
    \fontsize{30}{32}\selectfont 
    Reference Grammar\\
    \vspace{2cm}
    Kim Korsæth
}
\author{}
\date{\today}

%----TABLE OF CONTENTS---
%------------------------

\cftpagenumbersoff{part}
\cftsetindents{part}{1.5em}{1.5em}
\renewcommand\cftchapterfont{\textrm}



%--------MAIN DOC--------
%------------------------

\begin{document}


\maketitle

\newpage

\frontmatter

\chapter*{Preface}

\lang{} is a constructed language, and the speakers are entirely fictional. This grammar is a work of fiction.

THIS IS A WORK IN PROGRESS. Some sections may lack a lot of detail or be worded very poorly. I'm not good at translating my thoughts and ideas into words, and while I'm constantly refining and expanding the whole grammar, it'll be at my own (considerably slower) pace. I ask that you reserve some judgement until I can call this done, and hope that you can enjoy and get excited by what is here so far.

\newpage
\tableofcontents
\listoffigures
\listoftables
\newpage

\chapter{Abbreviations}

\begin{multicols*}{2}
    
\newcommand{\listabbrev}[2]{
\begin{minipage}[t]{0.17\columnwidth}
    \textbf{#1}
\end{minipage}
\begin{minipage}[t]{0.8\columnwidth}
    {#2}
\end{minipage}
\vspace{1ex}
}

\noindent\listabbrev{*}{unattested, ungrammatical, or unacceptable}
\listabbrev{›}{acts upon, e.g. 1›2, 3›3'}
\listabbrev{\textsc{act}}{actual aspect}
\listabbrev{\textsc{ani}}{animate}
\listabbrev{\textsc{antip}}{antipassive}
\listabbrev{C}{consonant}
\listabbrev{\textsc{der}}{derivational morpheme}
\listabbrev{\textsc{ds}}{different subject}
\listabbrev{\textsc{du}}{dual number}
\listabbrev{\textsc{fact}}{factual aspect}
\listabbrev{\textsc{hab}}{habitual aspect}
\listabbrev{\textsc{ina}}{inanimate}
\listabbrev{\textsc{inst}}{instrumental}
\listabbrev{\textsc{nob}}{null-object marker}
\listabbrev{\textsc{o, obj}}{object}
\listabbrev{\textsc{obv, '}}{obviative, e.g. 3'}
\listabbrev{\textsc{p, pl}}{plural number}
\listabbrev{\textsc{pct}}{punctual aspect}
\listabbrev{\textsc{prt}}{partitive}
\listabbrev{\textsc{s, sg}}{singular number}
\listabbrev{\textsc{sbj}}{subjunctive}
\listabbrev{\textsc{sr}}{switch reference}
\listabbrev{\textsc{V}}{vowel}
    
\end{multicols*}

\newpage

\mainmatter

\chapter{Phonology}

The phonemic inventory, at 14 contrasting segments, is considerably smaller than the world-average 27. Below is a table of all the phonemic consonants, as well as more detailed descriptions of each phoneme and its phonetic realizations below.

The French grammar had a very simple and inaccurate phonological analysis, where the voiced allophones of /p t k/ were considered a discrete set of phonemes.

\section{Consonants}

\begin{table}[ht]
    \centering
    \begin{tabular}{rccccc}
        \toprule
        & Labial & Alveolar & Palatal & Velar & Glottal \\
        \midrule
        Plosive & p & t & & k & \\
        Fricative & & s & & & h \\
        Lateral fricative & & ɬ & & & \\
        Nasal & m & n & & & \\
        Approximant & & & j & w & \\
        \bottomrule
    \end{tabular}
    \caption{Phonemic consonants of \lang}
\end{table}

% ɡ unicameral g here

\lang{} has 10 phonemic consonants.

% put here explanations for each phoneme along with initial, medial and final realizations along with any other special cases.

\subsection{Intervocalic voicing} \label{intervocalic}

It is a feature of all unvoiced segments except /ɬ/ and /h/ to become voiced between two voiced segments. The voiced segments must be directly adjacent, and clusters of unvoiced consonants are never voiced.

% examples here

\subsection{Initial fortition} \label{initfort}

The fricatives /s h ł/ undergo fortition to [t\bind{}s ʔ t\bind{}ɬ] word-initially.

% examples here

\section{Vowels}

\begin{table}[ht]
    \centering
    \begin{tabular}{lccc}
        \toprule
        & Front & Central & Back \\
        \midrule
        High & i & & u \\
        Mid & e & &  \\
        Low & & a & \\
        \bottomrule
    \end{tabular}
    \caption{Phonemic vowel inventory of \lang}
\end{table}

\lang{} contrasts only four vowel qualities. 

\paragraph{/a/} is a low central vowel, almost always pronounced like the cardinal vowel [a].

\paragraph{/e/} is a mid front vowel, usually pronounced like [e], but may be lowered to [ɛ]. Before /w/, it is realized as a schwa [ə].

\paragraph{/i/} is a high front vowel, pronounced like the cardinal vowel [i].

\paragraph{/u/} is a high back rounded vowel, usually pronounced like a near-back vowel [ʊ]. It can sometimes appear as low as [o].

\section{Syllables and moras}

When it comes to the segmental phonology of \lang{}, moras are a far more useful metric than syllables. 

Syllables consist of one or two moras, where the nucleus and onset count as one, and the coda counts as one. From there we can define any syllable containing one mora as a \textbf{light syllable}, and any syllable containing two moras as a \textbf{heavy syllable}. Words can contain as many light syllables as possible, but can't ever have more than two consecutive heavy syllables.

\begin{figure}[ht]
    \Tree [.σ [.μ C V ] [.μ C/V ] ]
    \caption{Basic moraic description of \lang{} syllables}
\end{figure}

The coda position may contain any consonant or a duplicate of the nucleus vowel.

\begin{table}[ht]
    \centering
    \begin{tabular}{>{\bfseries}lll}
        \toprule
        V & light syllable & \\
        CV & light syllable & \\
        VC & heavy syllable & \\
        VV & heavy syllable & \\
        CVV & heavy syllable & \\
        CVC & heavy syllable & \\
        \bottomrule
    \end{tabular}
\end{table}

\subsection{Restrictions}

/ju/ and /iw/ are in complementary distribution, same as /uj/ and /wi/

/Vji/ is rare and very frequently turns into /Vj/; /\%ijV/ into /\%jV/ has already happened recently.

\subsection{Semivowel codas and vowel length}

The coda position of the syllable can either be filled by a consonant or a vowel of the same quality as the nucleus to form a long vowel.

The approximants /j/ and /w/ can be interpreted as high vowels not occupying the nucleus position of the syllable, thereby forming a pseudo-minimal pair with /i/ and /u/ with the feature [±syl] distinguishing them. Underlyingly, both the vowels and approximants are |i| and |u|. According to this model, every |i| and |u| wants to become the nucleus of its own syllable, but are in most cases forced out of that position by a more salient vowel, or they must merge with an adjacent vowel; if they meet neither of these criteria, the moras are allowed to become syllabic and act like vowels. The reasoning behind this analysis is that any syllable with a high vowel and its equivalent approximant will merge to form one long vowel without a perceivable change in articulation between them. At the same time, the approximants may surface as vowels when not next to an existing vowel, for example where a morphological operation places either approximant next to a consonant without an adjoining vowel. Below is a list of examples involving /j/, demonstrated as going from the underlying moraic composition, then phonemic segmentation, then phonetic realization:

% put examples of /j/ here

\noindent The same happens with /w/:

% put examples of /w/ here

\newpage

\section{Orthography}

\begin{table}[ht]
    \centering
    \begin{tabular}{llll}
        \toprule
        Letter & Phoneme & Letter & Phoneme\\
        \midrule
        a & /a/ & n & /n/ \\
        e & /e\ttilde{}ə/ & p & /p/ \\
        h & /h/ & s & /s/ \\
        i & /i/ & t & /t/ \\
        k & /k/ & u & /u/ \\
        th& /ɬ/ & w & /w/ \\
        m & /m/ & y & /j/ \\
        \bottomrule
    \end{tabular}
    \caption{\lang{} orthography, devised by \$\$\$}
\end{table}

The phonemes /p t k s/ are represented by \sqbrack{b d g z} when they are voiced. /j/ and /w/ are represented by \sqbrack{i} and \sqbrack{u} respectively when following vowels or interconsonantally, but not intervocally.

There have been efforts to adopt Cree syllabics, but this has so far not seen widespread support.

\chapter{Verbs} 

% TOTALLY retain the move/feel/change/remain distinction in some way 

The verb is the biggest focal point of \lang{} morphosyntax. Most things are expressed within the verb phrase. The distinction between inflection and derivation is difficult to ascertain, and a lot of affixes can be either.

Since verbs can be derived from both verbal and nominal roots, no distinction is made between the two in this section. Instead, I will refer to stems as the nucleus of the morphological verb throughout this chapter.

The verbal morphology can be laid out as such:

\begin{table}[ht]
    \centering
    \begin{tabular}{>{\bfseries}lll}
        \toprule
        Slot & Sub-slot & Use \\
        \midrule
        -1 & & preverbal affixes \\
        0 & 0a & verb root \\
          & 0b & additional verb \\
          & 0c & derivational suffixes \\
        1 & 1a & -y suffix \\
          & 1b & detransitive suffixes \\
        2 & & final suffix \\
        \bottomrule
    \end{tabular}
\end{table}

The following sections will go through each group of affixes in the order they appear in this list.

\section{Preverbal affixes}

\begin{table}[ht]
    \centering
    \begin{tabular}{>{\bfseries}ll}
        \toprule
        ? & factual preverb \\
        ? & actual preverb \\
        ? & nonactual preverb \\
        ? & punctual preverb \\
        ? & defective preverb \\
        ? & jussive preverb \\
        ? & resumptive preverb \\
        \bottomrule
    \end{tabular}
    \caption{Preverbs in \lang}
\end{table}

\subsection{Factual preverb}

"indicative" for stative stems (nouns, adjectives)

habitual for active stems (durative, punctual)

\subsection{Actual preverb}

"indicative" for durative stems

punctual-turned-frequentative takes Actual preverb

\subsection{Nonactual preverb}

irrealis/subjunctive-ish, verbal noun-ish, must be used when verb is negated

\subsection{Punctual preverb}

"indicative" for punctual stems

\subsection{Defective preverb}

action halted before culmination, either unstarted (atelic) or unfinished (telic)

\subsection{Jussive preverb}

should, ought to, have to, must

\subsection{Resumptive preverb}

once more, resumes, repeats (once), NOT frequentative

\section{Verb stem}

\subsection{Punctual stems}

instantaneous events or actions

\subsection{Durative stems}

lasting events or actions

\subsection{Stative stems}

States, description, copula-ish

\subsection{Inchoative stems}

start of durative event

\subsection{Frequentative stems}

repetitive action

\subsection{Reduplication}

Two patterns of reduplication have been recorded, rightward duplication and full reduplication. The latter is more productive than the former.

\subsubsection{Rightward reduplication}

Rightward reduplication involves copying some part of the root onto the end of the root, and can take several different forms depending on the root's final syllable or syllables, shown in the table below.

\begin{table}[ht]
    \centering
    \begin{tabular}{lll}
        \toprule
        Root ending & Reduplicated & Example \\
        \midrule
        CV    & CVC(V)  & -pap `become large, grow' (< -pa `be large') \\
              &         & -wanin `deepen, sink' (< -wani `be deep') \\
              &         & -hahuhu `fall asleep' (< -hahu `sleep') \\
        (C)VC & CVCVC   & -kutut `about to notice' (< -kut `notice') \\
              &         & -hepasas `start to bend, buckle' (< -hepas `bend') \\
              &         & -newew `where to' (< -new `what') \\
        CV\longv{} & CV\longv{}CV & -tiidi `sit down' (< -tii `sit, be sitting') \\
              &         & -taata `split up, split in two' (< -taa `two, a pair') \\
              &         & -thameeme `go around' (< -thamee `return') \\
        (C)VCCV & (C)VCCVCV & -patwewe `set off' (< -patwe `go') \\
              &         & -untutu `demonstrate' (< -untu `lead') \\
        \bottomrule
    \end{tabular}
    \caption{Rightward reduplication patterns in \lang}
\end{table}

All the different patterns have in common that they only add one mora to the stem. The simplest process is for light monosyllables, where they take a duplicate of the onset consonant as a coda, or duplicate the entire last syllable if that onset is /h/. On closed syllables, the nucleus and coda are duplicate to form another syllable with the same coda. On long vowel syllables, the first mora of the syllable is duplicated. On multisyllabic roots ending in a light syllable preceded by a heavy syllable, the entire last syllable is reduplicated.

All stems formed by rightward reduplication fall into the category of inchoative stems, as they all describe some form of transition into a new state or action. On stative stems it marks meanings akin to `become X' or `turn into X', on durative stems it marks a meaning akin to `start to X', while on stems describing punctual events it suggests imminent action akin to `about to X'. The resulting inchoative stems usually end up punctual in nature, taking the punctual affix \textit{ki-}. Many inchoative stems are no longer productive and have taken on meanings that aren't interchangeable with their equivalent underived stems; a couple examples can be seen in the above table.

The exact lexical changes can vary. Some stems simply describe the initiaton of an activity: -kiiki `start to write' (< -kii `write'), -nuipapa `start to perform' (< -nuipa `perform a song/dance'), -tawawa `start to eat' (< -tawa `eat'), -apasas `set off on foot' (< -apas `travel far on foot, trek'). Other stems indicate imminent action: -kutut `about to notice' (< -kut `notice'), -matat `about to shriek' (< -mat `shriek'), -sayai `about to play with' (< -sai `play with'). These stems usually come from punctual actions, but some durative actions can take on a sense closer to this than the simple inchoative.

Others again have more complex relations. -hepasas `bend, yield, give out' implies a permanent or lasting change to the shape of a rigid object, while the original stem -hepas `bend' can be used for any non-permanent or reversible bending of a rigid or jointed object. -thameeme `go around, revolve around' has an adjacent meaning to -thamee `return'; it's understood to come from having been used in the sense of marking the point where a round trip turns back towards the starting location, akin to a thrown object reaching the apex of its arch.

\subsubsection{Full Reduplication}

Full reduplication of the root forms a frequentative stem. This involves duplicating the entire root with a ligature vowel if necessary to be phonotactically valid. On durative stems it marks that the action is either repeated or lengthened. No distinction is made between starting a task several times or continuing one task.

% example here

In contrast, on punctual actions it marks only repetition. 

% example here

Reduplicated stative stems translate to something like `just X, simply X' or in some cases `barely X'. In almost all cases, whatever is marked undershoots expectations in some way, to the surprise or relief of the relevant party, rarely disappointment.

% example here

This can also be used for affection or adoration towards someone or something.

% example here

\section{Final suffix}



Intransitive verbs take no person marking. Transitive verbs are marked for person agreeing with the syntactic pivot. The first person is used for the speaker or a group including the speaker. The second person is used for the addressee (whoever is being spoken to) in much the same way. The third person refers to anything not a speech act participant, including impersonal or circumstantial events, e.g. `it's raining'.

\chapter{Nominals}

While pure nouns exist, it is very common for nominals to be derived from verbs.

\section{Case}

\subsection{Absolutive}

least agentive argument takes absolutive, always. 

intransitive verbs take absolutive because the sole argument is the least agentive argument.

\subsection{Ergative 1}

marks the agent of transitive verbs where the action/change described can only come about as a result of someone's action. e.g. something can't be chewed by any other means than someone doing it, making the agent inherent to the action, and so the agent of the verb 'chew' takes the ergative 1 case. 

\subsection{Ergative 2}

marks the agent of transitive verbs where the action/change described can happen through various means, not just through deliberate action. e.g. something can fall apart on its own, or by inanimate forces of nature, but a volitive agent can come around and take it apart deliberately. Similarly, something can heat up on its own, or through external forces both volitive and non-volitive. As such, "take apart" and "warm up(caus.)" would take an ergative 2 agent

\section{Number}

-um?

\section{Pronouns}

There are personal pronouns and demonstrative pronouns. demonstrative pronouns also encode some pragmatic effect, figure out what.

\begin{table}[ht]
    \centering
    \begin{tabular}{lllll}
        \toprule
                             & Absolutive   & Ergative 1 & Ergative 2 & ? \\
        \midrule
        1st person sg.       & awa          &       &   & \\
        1st person pl.       &          &     &   & \\
        1-2. person incl.    & awattaa    &  &   & \\
        2nd person sg.       & igaw         &     &   & \\
        2nd person pl.       &        &   &   & \\
        3rd person ani. sg.  & he           &        &   & \\
        3rd person ani. pl.  &           &     &   & \\
        % 3rd person inan. sg. & simaa        &      &   & \\
        % 3rd person inan. pl. &       &   &   & \\
        \bottomrule
    \end{tabular}
    \caption{Personal pronouns of \lang}
\end{table}

\chapter{Syntax}


\section{The marked argument - Subject}

Always one marked argument. The marked argument is what the utterance is about, and once pragmatically established can be omitted from subsequent utterances. 

Intransitive verbs: one argument, it's automatically the marked argument

Transitive verbs: two arguments, least agentive argument P (patient) is the marked argument. -y on the verb switches this and makes most agentive argument A (agent) the marked argument

Ditransitive verbs: three arguments, least agentive argument T (theme) is the marked argument. -y on the verb switches this and makes more agentive argument R (recipient) the marked argument

Tritransitive verbs: lol lmao

\section{Simple clauses}

Most clauses are verb-final. 

\subsection{Intransitive}

SV

\subsection{Transitive}

verb last, marked argument first

\subsection{Ditransitive}

agent-recipient-theme-verb

\subsection{Causative}

causer-verb-causee-object 

\section{Parataxis}

things are just put next to each other for the most part, subordination does happen but not as often as expected

\end{document}